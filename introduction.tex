\setlength{\parskip}{5pt plus 1pt minus 1pt}
\section{Introduction}

Advanced real-time rendering applications often involve rendering large animated models using complex lighting and shading techniques. Depending of the relative complexity between the rendered geometry and the fragment shading algorithm, the rendering process is often bottlenecked at either the {\em vertex shader} or {\em fragment shader} stage. Scenes that fall into one of these categories are referred to as  {\em vertex-bound} and {\em fill-bound} scenes, respectively. Approaches have been proposed to reorder the triangles of a mesh so as to alleviate these bottlenecks.

In order to reduce vertex computation, the application can leverage the GPU's post-transform vertex caching mechanism, which stores the vertex shading output of a small set of recently processed vertices. When processing a particular vertex, recomputation can be avoided if the vertex has recently been processed by an adjacent triangle within the same hardware unit and thus is still cached. This encourages a triangle order with vertex reference locality (i.e., mesh triangles that share vertices should be close to each other in the index buffer). The average cache miss ratio (ACMR) of a particular triangle order measures the ratio between processed vertices and rendered triangles for a given caching scheme (usually a FIFO scheme is assumed). Generating triangle orders that reduce ACMR results in a significant improvement in rendering time for heavily vertex-bound scenes.

Scenes may also have very complex lighting and shading techniques, resulting in computationally intensive fragment shaders. In this case, reducing the number of fragments that need to be shaded can significantly reduce rendering time. When rasterizing triangles, GPUs apply {\em early-Z culling}, which performs depth testing prior to fragment shading. Thus, if the triangles happen to be processed in perfect front-to-back order, {\em none} of the occluded fragments will need to be shaded. In the worst case, when rendering in back-to-front order, {\em all} of the fragments need to be shaded, even those that are completely occluded by subsequent triangles. The overdraw ratio (OVR) of a triangle order refers to the ratio between the total number of fragments that passed the depth test and the number of visible fragments. An overdraw ratio of 1 is optimal and means no overdraw.

It has been shown that for heavily vertex-bound scenes, ACMR is directly proportional to rendering time, while for heavily fill-bound scenes, OVR is directly proportional to rendering time~\citep{Sander07}. An efficient triangle order has both low ACMR and low OVR. In this paper we propose a technique that, as in \cite{Nehab06} and \cite{Sander07}, compromises between these two objectives. However, unlike these previous techniques, our approach handles animated meshes. Since we are addressing keyframe animations where mesh connectivity does not change, ACMR is invariant to the animation. On the other hand, since vertices change their relative positions over the course of the animation, OVR can be significantly affected. Our algorithm generates a set of triangle orders that minimize OVR over the entire animation sequence, while still maintaining a low ACMR.